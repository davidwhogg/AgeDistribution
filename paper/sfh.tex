% This file is part of the AgeDistribution project.
% Copyright 2015 the authors.

\newcommand{\project}[1]{\textsl{#1}}
\newcommand{\sdss}{\project{\textsc{sdss-III}}}
\newcommand{\apogee}{\project{\textsc{apogee}}}
\newcommand{\kepler}{\project{Kepler}}
\newcommand{\tc}{\project{The~Cannon}}

\newcommand{\foreign}[1]{\textsl{#1}}
\newcommand{\etal}{\foreign{et~al.}}

\newcommand{\dd}{\mathrm{d}}
\newcommand{\given}{\,|\,}
\newcommand{\setof}[1]{\left\{{#1}\right\}}

\documentclass[12pt, preprint]{aastex}

\begin{document}

\title{A direct measurement of the star-formation history of the Milky Way disk}
\author{%
  Hogg,
  Ness,
  Rix,
  Martig,
  \emph{others}}
\date{\texttt{DRAFT --- 2015-08-13 --- NOT FOR DISTRIBUTION}}

\begin{abstract}
% Context
The star-formation history of the Milky Way has been inferred
through chemical evolution models, which depend on assumptions about
infall, outflow, and yields.
There are also more direct methods, that use stars (such as white
dwarfs and M-type dwarfs) that provide age indicators.
% Aims
Here we measure the star-formation history by a direct method, using
red-clump stars in the \sdss\ \apogee\ sample, for which we have age
estimates delivered by a data-driven model of stellar spectra (\tc).
The stellar sample is novel in its size (XX,000 stars), its spatial
extent (Galactocentric radii of 6 to 14~kpc), and its positional
accuracy (distances good to XX~percent).
% Methods
\tc\ is trained using stars observed by both \apogee\ and \kepler, for
which the latter gives (via asteroseismology) a reliable mass
estimate---and therefore age estimate (given current models of stellar
evolution).
\tc\ is applied to the full \apogee\ red-clump star sample, providing
masses good to 0.08~dex (and therefore ages good to 0.25~dex).
We model (deconvolve) the red-clump-star mass distribution as a function
of Galactocentric radius, and convert this into a star-formation history
using models of stellar evolution and the initial mass function.
% Results
We find ZZZ.
\end{abstract}

\section{Introduction}

Literature on chemical-evolution star-formation histories.

Literature on the local SFH (from, say, Hipparcos?)

Literature on M-type dwarfs, white dwarfs, rotation, anything else?

What \tc\ can do and has done.

What the red-clump sample can do and has done.

What are our objectives right now?

\section{Method and results}

Off-load all mass and age estimation to Ness \etal\ (2015).

The mass estimates have large scatter, so the observed mass
distribution is related to the true mass distribution by a substantial
convolution.
We infer the true mass distribution by forward-modeling the observed
mass distribution.
This is a kind of probabilistic deconvolution.

\begin{eqnarray}
  \left.\frac{\dd N}{\dd\ln M}\right|_{M,R}
  &=&
  \sum_{k=1}^{K} a_{k}(R)\,\delta(\ln M - \ln M_{k})
  \\
  \left.\frac{\dd N}{\dd\ln\tilde{M}}\right|_{\tilde{M},R}
  &=&
  \sum_{k=1}^{K} a_{k}(R)\,N(\ln\tilde{M}\given \ln M_{k}, \sigma^2)
  \\
  a_{k}(R)
  &=&
  \exp(\alpha_{k} + \beta_{k}\,[R - R_0])
  \quad ,
\end{eqnarray}
where...

\begin{eqnarray}
  D
  &\equiv&
  \setof{\ln\tilde{M_n}}_{n=1}^{N}
  \\
  \ln p(D)
  &=&
  \sum_{n=1}^N \ln\left.\frac{\dd N}{\dd\ln\tilde{M}}\right|_{\tilde{M_n},R_n}
  - \left<N\right>
  \\
  \left<N\right>
  &\equiv&
  \int \left.\frac{\dd N}{\dd\ln\tilde{M}}\right|_{\tilde{M},R}\,\dd\ln\tilde{M}
\end{eqnarray}
where... [SOMETHING IS WRONG, because the $R$ must be averaged over somehow.]

Introduce some terminology around age and time, and also mass of a
star and mass of a stellar population.
These are two key confusions for what follows.

We infer a star-formation history from a red-clump mass distribution
as follows:
\begin{eqnarray}
  \left.\frac{\dd N}{\dd\ln M}\right|_{M,R}
  &=&
  \left.\frac{\dd m}{\dd t\,\dd A}\right|_{t(M),R}
  \,\left|\frac{\dd t}{\dd\ln M}\right|_{M}
  \,\left.\frac{\dd N}{\dd m}\right|_{t(M)}
  \,\Delta A
  \\
  \left.\frac{\dd N}{\dd\ln M}\right|_{M,R}
  &\mbox{is}&
  \mbox{the deconvolved observable}
  \\
  \left.\frac{\dd m}{\dd t\,\dd A}\right|_{t(M),R}
  &\mbox{is}&
  \mbox{the star-formation rate (mass per time per area)}
  \\
  t(M)
  &\mbox{is}&
  \mbox{the age $t$ corresponding to mass $M$}
  \\
  \,\left|\frac{\dd t}{\dd\ln M}\right|_{M}
  &\mbox{is}&
  \mbox{the Jacobian that converts $\ln$ mass to age}
  \\
  \,\left.\frac{\dd N}{\dd m}\right|_{t(M)}
  &\mbox{is}&
  \mbox{the number of RC stars per mass formed, after time $t$}
  \\
  \,\Delta A
  &\mbox{is}&
  \mbox{the size of the relevant survey volume}
\end{eqnarray}
where...

\section{Discussion}

We found blah...

Issues with IMF and stellar evolution.

Issues with radial migration.

\acknowledgements
It is a pleasure to thank
  Maria Bergemann (MPIA)
  and
  Julianne J. Dalcanton (UW)
for useful discussions.

The SDSS [boilerplate here]

\newcommand{\arxiv}[1]{\href{http://arxiv.org/abs/#1}{arXiv:#1}}
\begin{thebibliography}{}\raggedright

\bibitem[Ness \etal(2015a)]{TheCannon}
Ness, M.~K. \etal, 2015a, ApJ, SOMETHING

\bibitem[Ness \etal(2015b)]{NessAges}
Ness, M.~K. \etal, 2015b, in preparation

\end{thebibliography}

\end{document}
